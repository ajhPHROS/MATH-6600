\documentclass[12pt]{article}
	
\usepackage[margin=1in]{geometry}		% For setting margins
\usepackage{amsmath}				% For Math
\usepackage{amsfonts}
\usepackage{fancyhdr}				% For fancy header/footer
\usepackage{graphicx}				% For including figure/image
\usepackage{cancel}					% To use the slash to cancel out stuff in work

%%%%%%%%%%%%%%%%%%%%%%
% Set up fancy header/footer
\pagestyle{fancy}
\fancyhead[LO,L]{MATH 6600 (MOAM)}
\fancyhead[CO,C]{Homework 3}
\fancyhead[RO,R]{\textbf{Due} Tuesday 10/14/25}
\fancyfoot[LO,L]{}
\fancyfoot[CO,C]{\thepage}
\fancyfoot[RO,R]{}
\renewcommand{\headrulewidth}{0.4pt}
\renewcommand{\footrulewidth}{0.4pt}
%%%%%%%%%%%%%%%%%%%%%%

\begin{document}

\noindent
Please submit your solutions to the following problems on Gradescope by \textbf{6pm} on the due date. Collaboration is encouraged, however, you must write up your solutions individually.

\bigskip

\noindent
\textbf{1) Orthogonal Projection Again.} Let $E:\mathbb{R}^N\rightarrow \mathcal{H}$ be a quasimatrix, whose linearly independent columns are elements of a Hilbert space $\mathcal{H}$ with inner product $\langle\cdot,\cdot\rangle$, i.e.,
$$
E = \begin{bmatrix}
| & & | \\
e_1 & \cdots & e_N \\
| & & |
\end{bmatrix}.
$$

\begin{itemize}

	\item[\textbf{(a)}] Given $f\in\mathcal{H}$, show that $\mathbf{c}\in\mathbb{R}^N$ minimizes $\|E\mathbf{c}-f\|$ if and only if $E^TE\mathbf{c}=E^Tf$, where $E^TE$ and  $E^Tf$ are the matrix and vector, respectively, whose components are
$$
(E^TE)_{ij} = \langle e_j, e_i\rangle, \qquad\text{and}\qquad (E^Tf)_j = \langle f, e_j\rangle.
$$

	\item[\textbf{(b)}] Is the Gram matrix, $E^TE$, from part (a) invertible? Explain your reasoning.

	\item[\textbf{(c)}] Verify that $E(E^TE)^{-1}E^Tf$ is the orthogonal projection of $f$ onto ${\rm span}(e_1,\ldots,e_N)$.

\end{itemize}

\bigskip

\noindent
\textbf{2) Best Dictionary Approximation.} Use the Chebfun system in MATLAB to compare the numerical accuracy of two formulas for best approximation of a function $f:[-1,1]\rightarrow\mathbb{R}$:
$$
\mathbf{c}_1 = (E_N^TE_N)^{-1}E_N^Tf, \qquad\text{and}\qquad \mathbf{c}_2 = R_N^{-1} Q_N^Tf.
$$
Here, $E_N = \begin{bmatrix} 1 & x & \cdots & x^N\end{bmatrix}$ is the quasimatrix of monomials up to degree $N$ and $E_N = Q_NR_N$ is the QR decomposition of $E_N$ with respect to the $L^2([-1,1])$ inner product.

\medskip
\noindent
\textbf{Note:} You may find the MATLAB code \texttt{demo01.m} on the course repository useful.

\begin{itemize}

	\item[\textbf{(a)}] Given $f(x) = 1/(1+20x^2)$, plot the relative error $\|E_N\mathbf{c}_1-f\|/\|f\|$ for dictionary sizes $N=10,20,30,\ldots,800$. Use a base-$10$ logarithmic scale for the relative error axis. 

	\item[\textbf{(b)}] Repeat the experiment in part (a) for $\|E_N\mathbf{c}_2-f\|$ and compare the error curves. 

	\item[\textbf{(c)}] Plot the condition numbers of the matrices $R_N$ and $E_N^TE_N$ for dictionary sizes $N=10,20,30,\ldots,800$. Use a base-$10$ logarithmic scale for the condition number axis. 

	\item[\textbf{(d)}] Interpret the error curves in parts (a) and (b) in light of your experiments in part (c).

\end{itemize} 

\bigskip

\noindent
\textbf{3) Interpolation.} Use the Chebfun system in MATLAB to compare the approximation accuracy of polynomial interpolants in (i) equally spaced points and (ii) Legendre points:
$$
\texttt{xi = linspace(-1,1,N).';} \qquad \text{and} \qquad \texttt{xii = legpts(N);}.
$$
Use a Legendre basis to form the generalized Vandermonde matrices for interpolation.

\begin{itemize}

	\item[\textbf{(a)}]  Given $f(x) = 1/(1+20x^2)$, plot the relative error $\|f-p_N\|/\|f\|$ in both interpolants for $N=10,20,30,\ldots,500$. Use a base-$10$ logarithmic scale for the relative error axis. 

	\item[\textbf{(b)}] Plot the condition number of the generalized Vandermonde matrices for both interpolants for $N=10,20,30,\ldots,500$. Use a base-$10$ logarithmic scale for the condition number axis. 

	\item[\textbf{(c)}] Interpret the error curves in part (a) in light of your experiments in part (b).

\end{itemize} 

\end{document}