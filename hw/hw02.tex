\documentclass[12pt]{article}
	
\usepackage[margin=1in]{geometry}		% For setting margins
\usepackage{amsmath}				% For Math
\usepackage{amsfonts}
\usepackage{fancyhdr}				% For fancy header/footer
\usepackage{graphicx}				% For including figure/image
\usepackage{cancel}					% To use the slash to cancel out stuff in work

%%%%%%%%%%%%%%%%%%%%%%
% Set up fancy header/footer
\pagestyle{fancy}
\fancyhead[LO,L]{MATH 6600 (MOAM)}
\fancyhead[CO,C]{Homework 2}
\fancyhead[RO,R]{\textbf{Due} Monday 9/29/25}
\fancyfoot[LO,L]{}
\fancyfoot[CO,C]{\thepage}
\fancyfoot[RO,R]{}
\renewcommand{\headrulewidth}{0.4pt}
\renewcommand{\footrulewidth}{0.4pt}
%%%%%%%%%%%%%%%%%%%%%%

\begin{document}

\noindent
Please submit your solutions to the following problems on Gradescope by \textbf{6pm} on the due date. Collaboration is encouraged, however, you must write up your solutions individually.

\bigskip

\noindent
\textbf{1) Normed vector spaces.} Denote the set of all polynomials on $[-1,1]$ with real coefficients, of any degree, by $\mathbb{P}$ and define a map $\|\cdot\|:\mathbb{P}\rightarrow[0,\infty)$ by $\|p\|=\sup_{-1\leq x\leq 1}|p(x)|$.

\begin{itemize}

	\item[\textbf{(a)}] Verify that $\mathbb{P}$ is a vector space over $\mathbb{R}$ and that the map $\|\cdot\|$ is a norm on $\mathbb{P}$.

	\item[\textbf{(b)}] Name one finite-dimensional subspace and one infinite-dimensional subspace of $\mathbb{P}$.

	\item[\textbf{(c)}] Show that $\mathbb{P}$ is \textbf{not} complete with respect to the norm $\|\cdot\|$. 

	\noindent
	\textbf{Hint:} Construct a Cauchy sequence of polynomials whose limit is not a polynomial.

	\item[\textbf{(d)}] Show that the limit of any Cauchy sequence in $\mathbb{P}$ is a continuous function on $[-1,1]$.

\end{itemize}

\noindent
\textbf{Note:} The space of continuous functions with the supremum norm is an example of a \textit{Banach space}, a complete normed space. While MATH 6600 focuses on Hilbert spaces and the role of orthogonality, Banach spaces also occupy an important place in applied analysis.


\bigskip

\noindent
\textbf{2) Chebyshev Series.} The Chebyshev polynomials provide a ``nonperiodic analogue" of Fourier series for $[-1,1]$. They are defined by $T_n(x) = \cos(n\arccos x)$, for $n=0,1,2,3,\ldots$.

\medskip
\noindent
\textbf{Hint:} The substitution $x=\cos\theta$ may be useful in completing some of the following exercises.

\begin{itemize}

	\item[\textbf{(a)}] Verify that $T_n(x)$ is a degree $n$ polynomial by showing that it satisfies a three term recurrence: $T_0(x)=1$, $T_1(x)=x$, and $T_n(x) = 2xT_{n-1}(x) - T_{n-2}(x)$ for $n\geq 2$. 

	\item[\textbf{(b)}] Show that $\langle p,q\rangle=\int_{-1}^1p(x)q(x)(1-x^2)^{-1/2}dx$ defines an inner product on $\mathbb{P}$. 

	\item[\textbf{(c)}] Show that the Chebyshev polynomials form an orthogonal basis for $\mathbb{P}$.

\noindent
\textbf{Note:} For the purpose of this question, you may interpret ${\rm span}\{T_k\}_{k=0}^\infty$ as the set of all \textit{finite} linear combinations of Chebyshev polynomials. This interpretation of span corresponds to a \textit{Hamel} basis for the infinite-dimensional vector space $\mathbb{P}$.

	\item[\textbf{(d)}] Given a continuous function $f:[-1,1]\rightarrow\mathbb{R}$, give a formula for the \textit{best approximation} to $f$ in ${\rm span}\{T_0,T_1,\ldots,T_N\}$ with respect to the norm induced by $\langle\cdot,\cdot\rangle$.

	\item[\textbf{(e)}] Based on your work in Homework 1, if $f$ has $n$ continuous derivatives, then what rate of decrease do you expect for the best approximation error as $N\rightarrow\infty$ in part (d)?

\end{itemize}

\noindent
\textbf{Bonus (Optional).} On homework 1, you may have noticed that the Fourier coefficients of some of the nonsmooth functions decayed at a faster rate than the upper bounds you derived. Can you derive a sharper upper bound for the decay rate of Fourier coefficients of a function whose $n$th derivative is piecewise continuous with finitely many points of discontinuity? 

\end{document}