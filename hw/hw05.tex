\documentclass[12pt]{article}
	
\usepackage[margin=1in]{geometry}		% For setting margins
\usepackage{amsmath}				% For Math
\usepackage{amsfonts}
\usepackage{fancyhdr}				% For fancy header/footer
\usepackage{graphicx}				% For including figure/image
\usepackage{cancel}					% To use the slash to cancel out stuff in work
\usepackage[dvipsnames]{xcolor}

%%%%%%%%%%%%%%%%%%%%%%
% Set up fancy header/footer
\pagestyle{fancy}
\fancyhead[LO,L]{MATH 6600 (MOAM)}
\fancyhead[CO,C]{Homework 7}
\fancyhead[RO,R]{\textbf{Due} Tuesday 11/11/25}
\fancyfoot[LO,L]{}
\fancyfoot[CO,C]{\thepage}
\fancyfoot[RO,R]{}
\renewcommand{\headrulewidth}{0.4pt}
\renewcommand{\footrulewidth}{0.4pt}
%%%%%%%%%%%%%%%%%%%%%%

\begin{document}

\noindent
Please submit your solutions to the following problems on Gradescope by \textbf{6pm} on the due date. Collaboration is encouraged, however, you must write up your solutions individually.

\bigskip

\noindent
\textbf{1) Final Project Proposal.} Select a \textit{method} or \textit{problem} to study for your final project, following the guidelines on the course repository. Please submit a two paragraph proposal that includes:
\begin{itemize}
    \item[(a)] A brief description of your topic, identifying a method or problem of choice (in plain language, no math needed unless you find it helpful). You may want to include your own motivation for choosing this topic.

    \item[(b)] Three peer-reviewed resources that can serve as a starting point for your self-study. These could be a textbook, a monograph, a review article in a reputable journal, etc. Informal resources such as blogs and online forums will \textit{not} be counted.

    \item[(c)] A preliminary description of a problem (if you are focusing on a method) or a method (if you are focusing on a problem) that you can use for the ``application" requirement in your report, with an explanation of your choice. For example, ``why is this problem suitable for your method of focus?"
\end{itemize}

\end{document}
